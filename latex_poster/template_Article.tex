\documentclass[]{article}

\usepackage{amsmath}

%opening
\title{}
\author{}

\begin{document}

\maketitle

\begin{abstract}

\end{abstract}

\section{What is a Neural Network?}

A Neural Network is a machine learning model roughly based on the makeup of the brain. A NN is comprised of a series of \textbf{layers}. Each of these layers is comprised of \textbf{neurons}. Each neuron has an output, known as an \textbf{activation}, which is simply a linear combination of the outputs (activations) of the previous layers.

The connections between the neurons are often called \textbf{synapses}. Each synapse has an associated value which is the coefficient in the linear combination that produces the neurons in the next layer. The associated values of all the synapses between two layers comprise a \textbf{weight matrix}.

\section{Making Predictions}

Above we have simple neural network, which takes in two inputs (features), and returns one output (prediction). Let's assume we have 3 samples. This means X is a 3x2 matrix. $W_1$ is a matrix of synapse coefficients. We can see that there are four synapses between the first two layers. $W_1$ is a 2x2 matrix, where each row corresponds to the synapses coming out of one of the input neurons. In this neural net:

\begin{gather}
	z_2 = X W_1\\
	\hat{y} = z_2 W_2
\end{gather}

The hyperbolic tangent function scales each activation from -1 to 1. We're done! This is now a neural net capable of making predictions.

\section{Activation functions}

Perhaps at this point you have astutely observed that currently our Neural Network is basically just a linear regression model, since the output is a linear combination of the inputs. It's time to introduce one of the most important features of neural networks: \textbf{nonlinearity}. Most neural network layers apply a activation function to their activations before passing on their output to the next layer. This transformed activation is known as the layer \textbf{activity}. All we have to do to make our network capable of predicting nonlinear relationships is update our equations:

\begin{equation}
	a_2 = \tanh (z_2) = \tanh (XW_1)
\end{equation}

\section{Training a Neural Network}

Making predictions with NNs is great, but it's not worth much unless we can improve our predictions. Gradient descent to the rescue!

\subsection{Gradient Descent and Neural Networks}
With random weights, a NN is pretty terrible at making predictions. To improve our model, we first need to quantify exactly how wrong our predictions are. We'll do this with a cost function. A cost function allows us to express exactly how wrong or "costly" our models is, given our examples.

A common way to compute an overall cost is to take each error value, square it, and sum the values. 

\begin{equation}
	J = \Sigma \frac{1}{2}(y- \hat{y})^{2}
\end{equation}

This cost is a function of two things - the input data, and the weights on the synapses. We can't change the data, so we'll improve the accuracy by modifying the weights!

Conceptually, what we'll be doing is computing derivative of the cost with respect to each weight matrix. We'll use these derivatives to compile a gradient, which we can then use to change the weights incrementally as our network improves!

\end{document}
